\documentclass{article}
\usepackage{ctex}
\usepackage[xcolor]{listings} % 导入代码高亮包
\usepackage{xcolor} % 导入颜色包
\usepackage{graphicx} % 导入图形包
\author{陈祎伟\\3230102357}
\title{Lab 2 : 流水线异常和中断设计}
\date{2025年3月9日}

\lstset{
    basicstyle=\ttfamily\footnotesize, % 基本样式,小号等宽字体
    numbers=left,                      % 行号位置
    numberstyle=\tiny\color{gray},     % 行号样式
    stepnumber=1,                      % 行号步长
    numbersep=5pt,                     % 行号与代码间距
    backgroundcolor=\color{white},     % 代码背景色
    showspaces=false,                  % 显示空格
    showstringspaces=false,            % 显示字符串中的空格
    showtabs=false,                    % 显示制表符
    frame=single,                      % 代码框
    rulecolor=\color{black},           % 框颜色
    tabsize=4,                         % 制表符宽度
    captionpos=b,                      % 标题位置
    breaklines=true,                   % 自动换行
    breakatwhitespace=false,           % 仅在空格处换行
    title=\lstname,                    % 显示文件名
    keywordstyle=\color{blue},         % 关键字颜色
    commentstyle=\color{brown},        % 注释颜色
    stringstyle=\color{red},           % 字符串颜色
    escapeinside={\%*}{*)},            % 特殊字符
    morekeywords={*,...},              % 额外关键字
    language=[x86masm]Assembler        % 定义asm语言
}

\begin{document}
\maketitle
\section{设计思路}
\subsection{CSRRegs.v文件}
CSRRegs 模块用于实现 CSR 寄存器的读写操作。\par
在原先 CSR 寄存器读写指令实现的基础上,增加了对于 mepc、mcause、mtval、mstatus 的旁路写操作。\par
增加相关wire信号输入后,具体实现如下:\par
\begin{lstlisting}[language=Verilog]
    else if (bypass_EN) begin
        CSR[0] <= mstatus_bypass_in;
        CSR[9] <= mepc_bypass_in;
        CSR[10] <= mcause_bypass_in;
        CSR[11] <= mtval_bypass_in;
    end
\end{lstlisting}

其中,mstatus\_bypass\_in、mepc\_bypass\_in、mcause\_bypass\_in、mtval\_bypass\_in 分别为 mstatus、mepc、mcause、mtval 的旁路写入信号,而 bypass\_EN 为旁路写入使能信号。\par

\subsection{ExceptionUnit.v文件}
ExceptionUnit 模块用于处理异常和中断。\par
首先连接 CSRRegs 模块,实现读取与写入:\par
\begin{lstlisting}[language=Verilog]
    assign csr_raddr = csr_rw_addr_in;
    assign csr_r_data_out = csr_rdata;
    assign csr_waddr = csr_rw_addr_in;
    assign csr_wdata = csr_w_imm_mux ? csr_w_data_imm : csr_w_data_reg;
    assign csr_w = csr_rw_in;
    assign csr_wsc = csr_wsc_mode_in;
\end{lstlisting}

定义寄存器信号 in\_trap 用于指示是否处于异常或中断状态。\par
当发生异常或中断时,将 bypass\_EN 置为 1,并将相关信息通过旁路传递给 CSRRegs 模块,具体实现如下:\par
\begin{lstlisting}[language=Verilog]
    if (some error type) begin
        bypass_EN <= 1;
        mcause_bypass_in <= MCAUSE_ILLEGAL_INST; // or other error code, according to the error type
        mepc_bypass_in <= (1) epc_cur(exception) or (2) epc_next(interrupt);
        mtval_bypass_in <= (1) einst(illegal instruction) or (2) eaddr(load/store address misaligned) or (3) 32'h0(others);
        mstatus_bypass_in <= {mstatus[31:13], 2'b11, mstatus[10:8], mstatus[3], mstatus[6:4], 1'b0, mstatus[2:0]}; // set MPP to 11, MPIE(mstatus[7]) to MIE(mstatus[3]), MIE(mstatus[3]) to 0
        in_trap <= 1; // set in_trap to 1
    end
\end{lstlisting}
其中,当发生中断时,还需判断 in\_trap 信号是否为 1,若为 1 则不再处理中断。\par
\begin{lstlisting}[language=Verilog]
    else if (interrupt && !in_trap) begin
\end{lstlisting}

完成异常或中断处理,进入 mret 指令时,将 in\_trap 置为 0,具体实现如下:\par
\begin{lstlisting}[language=Verilog]
    else if (mret) begin
        in_trap <= 0;
    end
\end{lstlisting}

最后完成时钟周期以及流水线寄存器的更新,\textbf{最初}具体实现如下:\par
\begin{lstlisting}[language=Verilog]
    assign PC_redirect = (illegal_inst || l_access_fault || s_access_fault || ecall_m || interrupt && !in_trap) ? mtvec :
                         (mret) ? mepc : 32'h0;
    assign redirect_mux = (illegal_inst || l_access_fault || s_access_fault || ecall_m || (interrupt && !in_trap) || mret) ? 1'b1 : 1'b0;
    assign reg_FD_flush = (illegal_inst || l_access_fault || s_access_fault || ecall_m || (interrupt && !in_trap) || mret) ? 1'b1 : 1'b0;
    assign reg_DE_flush = (illegal_inst || l_access_fault || s_access_fault || ecall_m || (interrupt && !in_trap) || mret) ? 1'b1 : 1'b0;
    assign reg_EM_flush = (illegal_inst || l_access_fault || s_access_fault || ecall_m || (interrupt && !in_trap) || mret) ? 1'b1 : 1'b0;
    assign reg_MW_flush = (illegal_inst || l_access_fault || s_access_fault || ecall_m || (interrupt && !in_trap) || mret) ? 1'b1 : 1'b0;
    assign RegWrite_cancel = (illegal_inst || l_access_fault || s_access_fault || ecall_m || (interrupt && !in_trap) || mret) ? 1'b1 : 1'b0;
    assign MemWrite_cancel = (illegal_inst || l_access_fault || s_access_fault || ecall_m || (interrupt && !in_trap) || mret) ? 1'b1 : 1'b0;
\end{lstlisting}

但在实际测试中,发现中断发生时,当前进行的指令未能正常完成,经分析是RegWrite\_cancel信号在interrupt时被错误地置为了1,导致了寄存器写入被取消,因此需要对RegWrite\_cancel信号进行修改,如下:\par
\begin{lstlisting}[language=Verilog]
    assign RegWrite_cancel = (illegal_inst || l_access_fault || s_access_fault || ecall_m || mret) ? 1'b1 : 1'b0;
\end{lstlisting}

其中,发生异常时,将 PC 重定向到 mtvec,同时将流水线寄存器清空,取消 RegWrite 和 MemWrite 操作。\par
发生中断时,将 PC 重定向到 mtvec,同时将流水线寄存器清空,取消 MemWrite 操作,但保留 RegWrite 操作。\par
进行 mret 指令时,将 PC 重定向到 mepc,同时将 in\_trap 置为 0,完成异常或中断处理。\par

经过仿真分析以及下板验证,代码实现正确,能够正常处理异常和中断。\par

\section{思考题}
\subsection{精确异常和非精确异常的区别是什么?}
精确异常要求硬件能够精确定位触发异常的指令,确保该指令之前的所有操作均已完成且状态生效,后续指令完全取消,系统可回滚至一致状态适合对可靠性要求高的通用处理器(如CPU)。\par
非精确异常则允许异常发生后系统状态存在不确定性(如后续指令可能已部分执行),无法精准定位异常点,需软件介入修复,硬件设计简化但牺牲了状态一致性,常见于高吞吐场景(如GPU并行计算)或异步中断处理。\par
两者的选择本质上是硬件可靠性(精确)与执行效率(非精确)的权衡。\par

\subsection{阅读测试代码,第一次导致 Trap 的指令是哪条?Trap 之后的指令做了什么?}
第一条导致 Trap 的指令是 ecall 指令。(PC = 0x70) \par
进入 Trap 后,依次将 mepc、mcause、mstatus、mtval 寄存器的值写入 x26、x27、x28、x29 寄存器中;
接着判断 x27 (mcause) 是否小于 0, 若小于 0 说明发生了中断,否则发生异常,异常时将 x26 (mepc) 加 4.
最后将 x26 (mepc) 的值写回 mepc 寄存器中,并调用 mret 指令,返回异常发生点,结束异常处理。\par

\subsection{如果实现了 U mode,并以 U mode 从头开始执行测试指令,会出现什么新的异常?}
访问 M-mode 受保护的 CSR 触发非法指令异常。\par
测试代码中,csr\_test 访问了一些 M-mode 级别的 CSR(如 mscratch、mtvec、mepc、mcause 等)。在 U-mode 运行时,尝试访问这些寄存器会触发 非法指令异常。\par
例如以下代码片段:\par
\begin{verbatim}
    csrrwi x1, mscratch, 0x10 # PC = 0x2C, CSR[8] = 0x00000010
    csrr x1, mscratch         # PC = 0x30, x1 = 0x00000010
    csrrw x2, mscratch, x6    # PC = 0x34, CSR[8] = 0xffff0000, x2 = 0x00000010
    csrr x1, mscratch         # PC = 0x38, x1 = 0xffff0000
\end{verbatim}


\subsection{为什么异常要传到最后一段即 WB 段后,才送入异常处理模块?可不可以一旦在某一段流水线发现了异常就送入异常处理模块?}
我认为可以不需要等到 WB 段后才送入异常处理模块,但需要谨慎处理。\par
当在对应阶段识别出异常时,可以立即将异常信息传递给异常处理模块,同时指明异常发生的具体位置;\par
对于发生异常之后的流水线阶段(之前的指令),则不进行 flush 等处理,使先前的指令继续执行,直到异常处理完成后再进行流水线寄存器的更新。\par
而对于发生异常之前的流水线阶段(之后的指令),则进行 flush 等处理,取消后续指令的执行,直到异常处理完成后再进行流水线寄存器的更新。\par
这样可以更快地响应异常,减少异常处理的延迟,提高异常处理的效率。\par

\end{document}